\documentclass{efd-lecture}

\begin{document}

\begin{frame}
  \titlepage{}
\end{frame}
\begin{frame}
  \frametitle{Outline}
  \tableofcontents{}
\end{frame}

\section{Introduction}

\section{JSON Schema}

\begin{frame}
  \frametitle{JSON Schema}
  \begin{itemize}
    \item JSON schema is JSON also
    \item The JSON document being validated or described we call the
      \textit{\color{YellowOrange} instance}, and the document containing the
      description is called the \textit{\color{GreenYellow} schema}.
  \end{itemize}
\end{frame}

\begin{frame}[fragile]
  \frametitle{Hello World}
  \begin{itemize}
    \item This accepts anything, as long as it’s valid JSON
      \mint[bgcolor=Black]{json}|{}|
    \item The most common thing to do in a JSON Schema is to restrict to a specific type. The \textit{\color{YellowOrange} type} keyword is used for that.
    \begin{minted}[bgcolor=Black]{json}
{ "type": "string" }
    \end{minted}
  \end{itemize}
\end{frame}

\begin{frame}[fragile]
  \frametitle{Declaring a JSON Schema}
  \begin{itemize}
    \item It's generally good practice to include it, though it is not required.
    \item The \textit{\color{YellowOrange} \$schema} keyword is used to declare
        that something is JSON Schema.
    \item Since JSON Schema is itself JSON, it's not always easy to tell when
        something is JSON Schema or just an arbitrary chunk of JSON.
  \end{itemize}
  \begin{minted}[bgcolor=Black]{json}
{ "$schema": "http://json-schema.org/draft-07/schema#" }
{ "$schema": "http://json-schema.org/draft/2019-09/schema#" }
  \end{minted}
\end{frame}

\begin{frame}[fragile]
  \frametitle{Declaring a unique identifier}
  \begin{itemize}
    \item It is also best practice to include an
      <code class="hl-orange">\$id</code> property as a unique identifier for
      each schema.
    \item For now, just set it to a URL at a domain you control, for example:
  \end{itemize}
  \begin{minted}[bgcolor=Black]{json}
{ "$id": "http://yourdomain.com/schemas/myschema.json" }
  \end{minted}
\end{frame}

\section{Swagger}

\section{AsyncAPI}

\begin{frame}
  \frametitle{Event-Driven Architectures}
  \begin{itemize}
    \item In most cases, Event-Driven Architectures (EDAs) are broker-centric
  \end{itemize}
\end{frame}

\begin{frame}[fragile]
  \frametitle{Hello World}
  \begin{minted}[bgcolor=Black]{yaml}
asyncapi: 2.2.0
info:
  title: Hello world application
  version: '0.1.0'
channels:
  hello:
    publish:
      message:
        payload:
          type: string
          pattern: '^hello .+$'
  \end{minted}
\end{frame}

\section{Go}

\section{Confluence}


\end{document}
