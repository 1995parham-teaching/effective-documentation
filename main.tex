\documentclass{efd-lecture}

\begin{document}

\begin{frame}
  \titlepage{}
\end{frame}
\begin{frame}
  \frametitle{Outline}
  \tableofcontents{}
\end{frame}

\section{Introduction}

\section{JSON Schema}

\begin{frame}
  \frametitle{JSON Schema}
  \begin{itemize}
    \item JSON schema is JSON also
    \item The JSON document being validated or described we call the
      \textit{\color{YellowOrange} instance}, and the document containing the
      description is called the \textit{\color{GreenYellow} schema}.
  \end{itemize}
\end{frame}

\begin{frame}[fragile]
  \frametitle{Hello World}
  \begin{itemize}
    \item This accepts anything, as long as it’s valid JSON
      \mint[bgcolor=Black]{json}|{}|
    \item The most common thing to do in a JSON Schema is to restrict to a specific type. The \textit{\color{YellowOrange} type} keyword is used for that.
    \begin{minted}[bgcolor=Black]{json}
{ "type": "string" }
    \end{minted}
  \end{itemize}
\end{frame}

\section{Swagger}

\section{Go}

\section{Confluence}


\end{document}
